% Created 2023-12-19 Tue 10:19
% Intended LaTeX compiler: pdflatex
\documentclass[11pt]{article}
\usepackage[utf8]{inputenc}
\usepackage[T1]{fontenc}
\usepackage{graphicx}
\usepackage{longtable}
\usepackage{wrapfig}
\usepackage{rotating}
\usepackage[normalem]{ulem}
\usepackage{amsmath}
\usepackage{amssymb}
\usepackage{capt-of}
\usepackage{hyperref}
\date{\today}
\title{}
\hypersetup{
 pdfauthor={},
 pdftitle={},
 pdfkeywords={},
 pdfsubject={},
 pdfcreator={Emacs 29.1 (Org mode 9.6.6)}, 
 pdflang={English}}
\begin{document}


\section{Gate constrains:}
\label{sec:org3f61f29}

\subsection{Multiplication constrain:}
\label{sec:orgf218bc2}

\begin{itemize}
\item \(r * s = n\)
\end{itemize}

\subsection{constraining \(r \neq 1\)}
\label{sec:org6d50148}
\begin{itemize}
\item Making sure that the prover doesn't fool the verifier by using 1 as a factor.
\item We force them to use r-1's inverse.
\item \((r-1)(r-1)^{-1} = 1 \Rightarrow  r * (r-1)^{-1} - (r-1)^{-1} = 1\)
\end{itemize}

\subsection{constraining \(0< r < 8\):}
\label{sec:org5d55729}
We need to make sure r \% p = r
otherwise prover can cheat because everything has an inverse in \(FF_p\)
We use Binary decomposition:
\begin{itemize}
\item \(r0 + 2 * r1 = r01\)
\item \(r01 + 4 * r2 = r\)
\item \(r0 * r0 - r0 = 0\)
\item \(r1 * r1 - r1 = 0\)
\item \(r2 * r2 - r2 = 0\)
\end{itemize}

\subsection{similarly constraining \(b\):}
\label{sec:orga5c2f3e}

\begin{itemize}
\item \(s0 + 2 * s1 = s01\)
\item \(s01 + 4 * s2 = s\)
\item \(s0 * s0 - s0 = 0\)
\item \(s1 * s1 - s1 = 0\)
\item \(s2 * s2 - s2 = 0\)
\item \(s * (s-1)^{-1} - (s-1)^{-1} = 1\)
\end{itemize}

\section{Universal way of summerizing constrains:}
\label{sec:orgef1d037}
\(Q_l*a + Q_r*b + Q_o*c + Q_m*a*b + Q_c = 0\)

\section{Constrain for factorization:}
\label{sec:orgdababe0}
\(Q_l*a + Q_r*b + Q_o*c + Q_m*a*b + Q_c = 0\)

\begin{center}
\begin{tabular}{rrrrrlllr}
Gate No & Q\textsubscript{l} & Q\textsubscript{r} & Q\textsubscript{o} & Q\textsubscript{m} & Q\textsubscript{c} & a & b & c\\[0pt]
\hline
1 & 0 & 0 & 0 & 1 & -n & r & s & 0\\[0pt]
\end{tabular}
\end{center}

\section{Constrain table \(a\)}
\label{sec:orga0851f0}
\(Q_l*a + Q_r*b + Q_o*c + Q_m*a*b + Q_c = 0\)

\begin{center}
\begin{tabular}{rrrrrrllr}
Gate No & Q\textsubscript{l} & Q\textsubscript{r} & Q\textsubscript{o} & Q\textsubscript{m} & Q\textsubscript{c} & a & b & c\\[0pt]
\hline
2 & 1 & 2 & -1 & 0 & 0 & r0 & r1 & r01\\[0pt]
3 & 1 & 4 & -1 & 0 & 0 & r01 & r2 & r\\[0pt]
4 & -1 & 0 & 0 & 1 & 0 & r0 & r0 & 0\\[0pt]
5 & -1 & 0 & 0 & 1 & 0 & r1 & r1 & 0\\[0pt]
6 & -1 & 0 & 0 & 1 & 0 & r2 & r2 & 0\\[0pt]
7 & 0 & -1 & 0 & 1 & -1 & r & (r-1)\textsuperscript{-1} & 0\\[0pt]
\hline
\end{tabular}
\end{center}

\section{Constrain table  \(b\)}
\label{sec:orgbae5eb4}
\(Q_l*a + Q_r*b + Q_o*c + Q_m*a*b + Q_c = 0\)

\begin{center}
\begin{tabular}{rrrrrrllr}
Gate No & Q\textsubscript{l} & Q\textsubscript{r} & Q\textsubscript{o} & Q\textsubscript{m} & Q\textsubscript{c} & a & b & c\\[0pt]
\hline
8 & 1 & 2 & -1 & 0 & 0 & s0 & s1 & s01\\[0pt]
9 & 1 & 4 & -1 & 0 & 0 & s01 & s2 & s\\[0pt]
10 & -1 & 0 & 0 & 1 & 0 & s0 & s0 & 0\\[0pt]
11 & -1 & 0 & 0 & 1 & 0 & s1 & s1 & 0\\[0pt]
12 & -1 & 0 & 0 & 1 & 0 & s2 & s2 & 0\\[0pt]
13 & 0 & -1 & 0 & 1 & -1 & s & (s-1)\textsuperscript{-1} & 0\\[0pt]
\hline
\end{tabular}
\end{center}

\section{Trace Polynomial}
\label{sec:orgf389178}
Summerizing all of our constrains in one polynomial:

\(Q_l(x)*a(x) + Q_r(x)*b(x) + Q_o(x)* c(x) + Q_m(x)*a(x)*b(x) + Q_c(x) = 0\)
\end{document}